\section{Upstreaming the code}\label{sec:upstreaming}
The results presented above are based on an implementation that localises all
changes to the sch\_cake qdisc itself. This involves the \textit{mq-cake}
initialisation code walking the qdisc tree to find its own siblings to
initialise the cross-qdisc synchronisation data structure.

While this works well to evaluate the concept, this approach is not appropriate
for inclusion into the mainline Linux kernel (\textit{upstreaming}). So when
proposing these changes for inclusion into the kernel, we aim to include a
proposal for a better API for this shared state.

Specifically, we propose introducing a generalised notion of shared qdisc state
that will be managed by the parent qdisc (in this case the \textit{mq} qdisc).
This shared state will be allocated and freed by the parent \textit{mq}
instance, and is passed to the sub-qdisc when it is attached to the parent. The
shared state is keyed to a qdisc module owner, which means there will be one
instance of shared state across all identical qdiscs attached to each
\textit{mq} instance.

To use this state, a qdisc module only needs define how much memory is needed by
its shared data structure (by setting a \textit{shared\_size} parameter in its
qdisc operations parameters), and define a function to receive the assignment
from the parent, and (optionally) an initialisation function that is called the
first time the shared state is allocated for a given parent instance.

The assignment function will be called whenever the qdisc is attached to a
parent that manages shared state, and will contain a pointer to an object of the
size defined by the qdisc module. The child qdisc can assign this pointer to its
internal state and use it while running for any cross-qdisc operations. When the
sub-qdisc is detached, it will be notified of removal of the shared state, so it
can detach itself.

It is up to the qdisc module itself to manage concurrency across multiple
instances accessing the shared state while running, which is no different from
the implementation we have used for the evaluations in this paper. This also
means that using this shared state API does not change the functioning of the
multiqueue synchronisation algorithm itself, only the initialisation code used
to setup the data structures it uses to do its work.
