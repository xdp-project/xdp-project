% -*- TeX-engine: default; -*-
\documentclass[sigconf]{acmart}
\usepackage[font=footnotesize]{subcaption}
\usepackage[utf8]{inputenc}
\usepackage[T1]{fontenc}
\usepackage{url}
\usepackage{booktabs}
\usepackage{xcolor}

\setcopyright{none}
\acmConference[LPC '18 Networking Track]{Linux Plumbers Conference 2018 Networking Track}{Nov 13--14,
2018}{Vancouver, British Columbia}
\acmISBN{}
\acmDOI{}
%\widowpenalty=100
%\clubpenalty=100
%\brokenpenalty=100

\begin{document}
\title{XDP -- challenges and future work}
\author{Jesper Dangaard Brouer}
\affiliation{%
  \institution{Red Hat}}
\email{brouer@redhat.com}

\author{Toke Høiland-Jørgensen}
\affiliation{%
  \institution{Karlstad University}}
\email{toke@toke.dk}


\captionsetup{font+=small}


\begin{abstract}
XDP already offers rich facilities for high performance packet
processing, and has seen deployment in several production
systems. However, this does not mean that XDP is a finished system; on
the contrary, improvements are being added in every release of Linux,
and rough edges are constantly being filed down. The purpose of this
talk is to discuss some of these possibilities for future
improvements, including how to address some of the known limitations
of the system. We are especially interested in soliciting feedback and
ideas from the community on the best way forward.

The issues we are planning to discuss include, but are not limited to:

\begin{itemize}
\item User experience and debugging tools: How do we make it easier for people
  who are not familiar with the kernel or XDP to get to grips with the system
  and be productive when writing XDP programs?

\item Driver support: How do we get to full support for XDP in all drivers? Is
  this even a goal we should be striving for?

\item Performance: At high packet rates, every micro-optimisation counts. Things
  like inlining function calls in drivers are important, but also batching to
  amortise fixed costs such as DMA mapping. What are the known bottlenecks, and
  how do we address them?

\item QoS and rate transitions: How should we do QoS in XDP? In particular, rate
  transitions (where a faster link feeds into a slower) are currently hard to
  deal with from XDP, and would benefit from, e.g., Active Queue Management
  (AQM). Can we adapt some of the AQM and QoS facilities in the regular
  networking stack to work with XDP? Or should we do something different?

\item Accelerating other parts of the stack: Tom Herbert started the discussion
  on accelerating transport protocols with XDP back in 2016. How do we make
  progress on this? Or should we be doing something different? Are there other
  areas where we can extend XDPs processing model to provide useful
  accelerations?
\end{itemize}

\end{abstract}


\maketitle

\section{Introduction}%
\label{sec:introduction}

\bibliographystyle{ACM-Reference-Format}
\bibliography{xdp-future}

\end{document}
